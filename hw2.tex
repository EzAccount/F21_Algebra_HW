\documentclass{amsart}
\usepackage{amsmath}
\usepackage{amssymb}
\usepackage{amsthm}
\usepackage{tikz-cd}
\def\exp{{\rm exp}}
\def\R{\mathbb{R}}
\def\Z{\mathbb{Z}}
\def\Q{\mathbb{Q}}
\def\C{\mathbb{C}}
\def\N{\mathbb{N}}
\def\T{\mathbb{T}}
\def\Y{\mathbb{Y}}
\def\Sympl{\rm Sympl}
\def\Fix{\rm Fix}
\author{M. Tikhonov}
\title{Homework 2, MATH7751}
\setcounter{section}{2}
\newtheorem*{claim}{Claim}
\begin{document}
\maketitle
\subsection{}

Since rings are commutative, it's sufficient to check that the set is left-ideal.

(1) Pick arbitary $a \in I \cap J$ then, we have for $\forall r \in R$:
$$a \in I \iff ar \in I$$
$$a \in J \iff ar \in J$$

which, by definition of intersection, means that $ar \in I \cap J$. Thus $I \cap J$ is ideal.

(2) Assume $a = ij$, where $i \in I, j \in J$ and $ij$ doesn't belong to $I\cap J$. But since $I$ is ideal for $\forall j \in J$ $ij \in I$. But $J$ is an ideal too, so $\forall j \in J$ $ji \in J$, so by definition of intersection $ij \in IJ$ for all $i \in I, j \in J$.

(3) Let $f: R \to R/ IJ$. $\forall a \in I \cap J$:  $f(a)$ is nilpotent.
Since $a \in I \cap J$, $a^2 \in IJ$, then by definition $f(a^2) = 0$.

(4) $I + J = R$ then $IJ = I \cap J$

By (2) we have $IJ \subseteq I \cap J$, so we only need to show $I \cap J \subseteq IJ$.

Assume $x \in I \cap J$. Since $I+J=R$ we know that $\exists a \in I, b \in J$,s.t. $a+b = e$. But then $x = xa + xb$, so $xa, xb \in IJ$. But then $x \in IJ$.

\subsection{}

Let $f: R \to S$ be a ring homomorphism.

(1) Preimage of ideal is an ideal. 

Consider chain of injections $R \to S \to S/N$, where $N$ is an ideal. Let $g = f \circ \pi$, where $\pi$ is natural homomorphism $S \to S/N$. 

But $\ker g = f^{-1} (N)$. 

\begin{claim} Kernel of homomorphism is ideal.
\end{claim}
    Let $f: R \to R'$. Zero is in the kernel, so it's not empty. Let $u,v \in \ker f, r\in R$, then
    $$ f(ru) = f(r) f(u) = 0$$
    and 
    $$ f(u - v) = f(u) - f(v) = 0$$


(2) Image of surjective homomorphism is ideal.

Let $I$ be an ideal in $R$. Let $f: R \to S$ and we need to show that $J = f(I)$ is an ideal in S. By definition of ideal, it should be closed under addition and multiplication on elements of original ring.

Pick $a, b \in J$. Then there $\exists a', b' \in I$, s.t.
$$ f(a') = a, f(b') = b$$
We have:
$$a+b = f(a') + f(b') = f(a' +b')$$
by definition of homomorphism. Since $I$ is an ideal, $a' + b' \in I$, thus $f(I)$ is closed under addition.

Since $f: R \to S$ is surjective, $\exists r \in R$, s.t. $f(r) = s, \forall s$. Then by definition of homomorphism we have:

$$
    sa = f(r) f(a') = f(ra')
$$
But since $I$ is an ideal the product is in the ideal; thus $f(I)$ is closed under multiplication by a constant from $S$, which finished the proof that $f(I) \subset S$ is an ideal in $S$.

Let me now show that surjectivity is required for this claim to hold. Consider a ring that can be embedded to a field, say $\Z \subset \C$. Since $\C$ is a field it only has two ideals, zero and whole ring. But $\Z$ has non-trivial ideas (even numbers for example). Under the canonical (embedding) homomorphism thoose ideals map to non-ideal subsets ($2 \cdot 1/2 \notin 2Z$).

\subsection{} 
(1) Show that every nonzero ideal of $Z[i]$ has nonzero integer.
Consider $I$ to be non-zero ideal, then $\exists z = m + ni \in I$ with $m\cdot n \neq 0$. Let $\bar{z} = m-ni$; since $m,n \in Z$ we know that $\bar{z} \in Z[i]$. Then since $I$ is an ideal $z \bar{z} = m^2+n^2$ should be in ideal. Then since $(m,n) \neq (0,0)$ it's a non-zero integer element.

(2) Identify $Z[i] / (2+i)$. 

$Z[i]/(2+i) \simeq  Z[i]/(i^2+1, i+2) \simeq Z/(4+1) \simeq Z/5$

\subsection{}
Direct product of rings $R \times S = T$ is a ring.

Let $(a,b), (c,d), (e,f)$ be elements in T.

1. Since $R \times S$ is product of abelian groups, its abelian. 

2. Direct product of monoids $(R, *_R), (S, *_S)$ is a monoid $(T, \cdot)$. Moreover, it's commutative:
$$(a,b) \cdot (d,e) = (a *_R d ,b *_S e) = (d *_R a, e *_S b) = (d,e) \cdot (a,b)$$

3. 
\begin{align*}
    (a,b) \cdot [(c,d) + (e,f)] &= (a,b) \cdot (c+e, d+f) \\ &= ([a *_R c] + [a *_R e], [b *_S d] + [b *_S f])\\
     &= (a,b) \cdot (c,d) + (a,b) \cdot (e,f)
\end{align*}

\subsection{}

1. Consider $(1-e)^2 = 1 - 2e + e^2$. If $e$ is idenpotent, then $e^2=e$, so $(1-e)^2 = 1-e$, thus it's also idenpotent. And $e (1-e) = e - e^2 = e -e = 0$.

2. Assume $r \in eR$ be an element in ideal. Since it's generated by $e$, $r = a * e$ for some a. Then we have $a * e * e = a * e$, since $e*e = e$.

3. 
Consider multiplication by indempotent. It's surjective ring homomorphism. 
Proof:
Assume $ea \in (e)$
$$f(ea) = eea = ea$$
So $f$ is surjective.

For arbitary $x,y \in R$
$$ f(x+y) = e(x+y) = ex + ey = f(x) + f(y)$$
$$ f(xy) = exy = e^2 xy = ex e = f(x) f(y)$$

Clearly $(1-e) \subset \ker f$
Suppose $a \in \ker (f)$, i.e. $ea = 0$
$$ a = ea + (1-e)a = (1-e)a$$
So $\ker (f) \subset (1-e)$
Let $R_1 = (e), R_2 = (1-e)$. Define $f(a) : R \to R_1 \times R_2, a\mapsto (ea, (1-e)a)$.

Assume $a \in ker f$, then:
$ea = (1-e) a =0$. So $a = ea + (1-e)a = 0$and it follows that $a = 0$
Thus $f$ is injective. But it's also surjective.

Thus we are done.

\subsection{} 

Show that evaluation at point is surjective homomorphism from polynomials to ring. Fix point $c$.

For any $n \in \N$ $c^n \in R$ since $R$ is closed under multiplication. Since polynomial is finite sum and $R$ is closed under addition $f(c) \in R$. Denote it as $y$.

Since $y \in R$ let $f(x) = y \in R[x]$. Since we can find such polynomial for any $c$, it's surjective.



Show that kernel is principle ideal $(x-c)$. 

Clearly $(x-c) \in \ker$. Therefore the ideal generated by it is contained in kernel. Pick an arbitary $p(x) \in \ker$. Then:
$$ p(x) = (x-c) q(x) + r$$
for some $r \in R, q(x) \in R[x]$.
Apply evaluation to both sides, we have:

$$ 0 = (c-c) q(x) + r$$
So $p(x) = (x-c) q(x)$ lies in the ideal generated by $(x-c)$. Thus it generates entire kernel.

\subsection{}

Consider a map f $R \to R/I x R/J$ defined as $r \to (r+I, r+J)$. The kernel is $I \cap J$, so there's an injective homomorphism
$$R/ (I \cap J) \ to R/I \times R/J$$

On the other hand $f$ is surjective: consider $r,s \in R$. We can write $r = r_i + r_j$, $s = s_i + s_j$ where $r_i, s_i \in I$, $r_j, s_j \in J$. 

We have $ r_i + s_j - r = s_j - r_j \in J$ and $r_i + s_j - s \in I$, thus $r_i + s_j + I = s + I$ and $r_i + s_j + J = r + J$.

So it's surjective.

So we know:

$$ R / (I \cap J) \simeq R/I \times R/J$$

but by first problem $I \cap J = IJ = 0$ so $R / {0} = R \simeq R/I \times R/J$. 

\subsection{}

Show that $y^2 - x^3$ is principal ideal of $g(x,y) \mapsto g(t^2, t^3)$.

Clearly $y^2-x^3$ generates an ideal that is in kernel. 

Let $m, n$ be arbitary integers and let $n \geq 2$
we have:
$$ x^m y^n = - (x^3 - y^2) x^m y^{n-2} + x^{m+3} y^{n-2}
$$
Then, repeating the process:

$$ x^m y^n = (x^3-y^2) p(x,y) + q_0 (x) + q_1 (x) y$$

Thus, for arbitary $f(x,y)$ we can write:
$$
  f(x,y) = f_0(x) + f_1 (x) y + (x^3-y^2) g(x,y)
$$
where $f_0, f_1 \in C[x]$ and $g(x,y) \in C[x,y]$.

We have 
$$ 0 = f(t^2, t^3) = f_0 (t^2) + f_1 (t^2)t^3$$
Note that that $f_0 (t^2)$ only contain even terms, while $f_1(t^2) t^3$ only contains odd terms. Thus $f_0 (t^2) = 0, f_1(t^2)=0$, or $f_1 (x) =0$, $f_0 (x) = 0$.
So $f(x,y) = g(x,y) (x^3-y^2)$

Thus it lies inside ideal generated by $x^3-y^2$.

So we have $g(x) = t^2$ and $g(y) = t^3$. So a polynomial $f = \sum c_{ij} x^i y^j$ maps to $\sum c_{ij} t^{2i+3j}$. 

So any polynomial with zero cooeficient in front of $t$ can be represnted. $\text{Im} (g) = R + t^2 R[t]$

\end{document}