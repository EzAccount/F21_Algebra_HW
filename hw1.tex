\documentclass{amsart}
\usepackage{amsmath}
\usepackage{amssymb}
\usepackage{amsthm}
\usepackage{tikz-cd}
\def\exp{{\rm exp}}
\def\R{\mathbb{R}}
\def\Z{\mathbb{Z}}
\def\Q{\mathbb{Q}}
\def\C{\mathbb{C}}
\def\N{\mathbb{N}}
\def\T{\mathbb{T}}
\def\Y{\mathbb{Y}}
\def\Sympl{\rm Sympl}
\def\Fix{\rm Fix}
\author{M. Tikhonov}
\newtheorem{claim}{Claim}
\begin{document}

1. 
$$ -1a + a = -1a + 1a = (-1 + 1) a = 0a = 0$$

Now if $a$ is invertable, $a^{-1} \in R$, but then $-a^{-1} \in R$. Consider then:
$$(-a ^{-1}) (-a) = a^{-1} a = e$$

2. 
for $\forall x \in R$:

a)$$ x + x = (x+x)^2 = x^2 + x^2 + x^2 + x^2 = x + x + x + x$$
thus $x+x=0$

let $y \in R$


b) $$ x + y = (x+y)^2 = x^2 + xy + yx + y^2 = x + xy + yx + y$$
so 
$xy + yx = 0$, then by a) $xy = yx$

thus R is $F_2$ - algebra 

3. We need to show that $(R, \cdot)$ is a commutative semigroup and that operations respect distributivity.

$\forall a,b,c \in R$:\\

\begin{align*}
    &1. (ab) c = 0 = a (bc)\\
    &2. (ab) = 0 = (ba) 
\end{align*}


Since multplication is commutative, we only need to check one-sided distributivity:

$$3. a (b+c) = 0 = 0 + 0 = (ab) + (ac)$$

b) Now assume ring has unity.

$e \cdot e = e$ by definition of identity \\

$e \cdot e = 0$ by definition of $\cdot$ \\

Thus $e = 0$.\\

R has to be trivial, since $a = a \cdot e = 0$

4. Inheret associativity from the ring. Morever since $e=e \cdot e$ identity is containted.\\

By definition of unit, every element of $R^{\times}$ has inverse. Moreover since $a a^{-1} = a^{-1} a$ the inverse is also unit. Thus for every element of $R^{\times}$ we have inverse.\\

Now pick $a_1, a_2 \in R^{\times}$. We know:

$$a_1 a_2 a_2^{-1} a_1 = a_1 a^{-1} = 1$$
and

$$a_2^{-1} a_1^{-1} a_1 a_2 = a_2^{-1} a_2 = 1$$

so $$a_1 a_2 \in R^{\times}$$

\begin{claim}
    $\Z \simeq \Z_2$
\end{claim}
$\{-1, 1\} \subseteq \Z^{*}$ trivially, since $\{-1,1\}$ are always units.

Now let $ab = 1$.  WLOG assume $|a| \geq |b| > 0$. $1 = |ab| \geq |a|$ so $a \in \{-1 , 1\}$

\begin{claim}
    $Z[i]^* = \{-1, 1, i, -i \}$ 
\end{claim}
Clearly $\Z[i] \subseteq \C$, so we can inheret norm from $\C$. Assume $ab = 1$, then:

For $\| x + iy \| = x^2 + y^2$. So if $x,y \in \Z$ then $\|x + i y \| \in \Z$
$1 = \|ab\| = \|a\| \|b\|$ then $\|a\| = 1 / \|b\|$. So $\| b \| = 1$.

Solving $x^2 + y^2 = 1$ in integers gives $(\pm 1,0), (0, \pm 1)$ thus the statement.

Bonus: $\Z (\sqrt{2}) = \Z[x] / (x^2 - 2 )$. Let $i$ be inclusion $Z[x] \in R[x]$. 
Consider $f$ to be canonical homomorpshim  $Z[x] \mapsto Z[x] / (x^2-2)$. Induce the inclusion to quiotent rings. But $R[x] / (x^2-2) \simeq R$ since $x^2 - 2$ is reducable over $R$.

Clearly $(1+\sqrt{2})^n$ is unit. Following the same argument norm of unit $a^2 - 2b^2$ should be $1$ due to multiplicativity. Only solutions to that equations are given by continued fraction expansion of $\sqrt{2}$.


5. By reminders theorem $\text{mod} n$ split $\Z$ into disjoing partition.
Since we have bijection to partition  it defines a equivalence relationship.

Let $a_1 \simeq a_2$, $b_1 \simeq b_2$ -- meaning $a_1 = q_1 n + r$, $b_1 = q'_1 n + r'$, $a_2 = q_2 n + r$, $b_2 = q'_2n + r$

Consider $a_1 + b_1 - a_2 - b_2 = q_1 n + q'_1 n - q_2 n - q'_2 n \simeq 0$ 

So:

$$(a_1 + b_1) \simeq (a_2 + b_2)$$

Consider $a_1 b_1 - a_2 b_2 = (q_1 n + r)(q'_1 n + r') - (q_2 n + r)(q'_2 n + r') = n(q_1 q'_1 n + q_1 r' + q'_1 r - q_2 q'_2 n - q_2 r' - q'_2 r) \simeq 0$

Thus $a_1 b_1 \simeq a_2 b_2$

6.
Assume $a \neq 0$. 
Consider $\phi_1: x\to ax$. Since $a$ is not a zero devisor, $ax = ay$ means $x = y$ since it's left cancelable. So it's injective. Any finite set injective automorphism is surjective. Then each element has inverse.
So it's a division ring, but finite division rings are fields.

7. $\text{det} A = 0$ means that $\exists v$ s.t. $Av = 0$ and $\exists w$ s.t. $wA = 0$.

Now let B be a matrix with $b_{jk} = v_j w_k$.. Every row of such matrix is $w$ up to scalar multiplication and every column is $v$ upto scalar multiplication.
Thus $BA = AB = 0$

8. $x$ is nilpotent, then $\exists n$ s.t. $x^n = 0$.

a) If $r=0$ it's trivial. Assume not, then $(rx)^n = r^n x^n = r^n 0 = 0$
b) Consider $1+t \in R[[x]]$ and a natural map $f: R[[x]] \to R: x \mapsto t$ for nilpotent element $t$, since for nilpotent element series terminates at $n$.


Then $(1+x)^{-1} = \sum_1 ^ \infty (-1)^n x^n$ as formal power series. With natural map we have $(1+x)^{-1} \mapsto t^{-1}$.




\end{document}