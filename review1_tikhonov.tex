\documentclass{amsart}
\usepackage{amsmath}
\usepackage{amssymb}
\usepackage{amsthm}
\usepackage{tikz-cd}
\def\exp{{\rm exp}}
\def\R{\mathbb{R}}
\def\Z{\mathbb{Z}}
\def\Q{\mathbb{Q}}
\def\C{\mathbb{C}}
\def\N{\mathbb{N}}
\def\T{\mathbb{T}}
\def\Y{\mathbb{Y}}
\def\Sympl{\rm Sympl}
\def\Fix{\rm Fix}
\author{M. Tikhonov}
\begin{document}

\subsection{}

Let $T$ be injective map and $\{ v \}_n$ linearly independent, i.e. $\sum \alpha_i v_i = 0$ only if $\alpha_i = 0$ for each $i$.
Assume images are not linearly independent. Then there are scalaras $\beta_i$ with $\prod \beta_i \neq 0$ s.t.
$$ 0 = \sum \beta_i T (v_i) = T(\sum \beta_i v_i)$$
where we've used that $T$ is linear. Since $T$ is injective, $\sum \beta_i v_i = 0$, which contradicts the defenition of linear independent. Thus injective linear maps perserve linear independence.

\subsection{
}

\begin{claim}
    $dim (P_{n+1} (\R)) = n+2$
\end{claim}

\begin{proof}
    Let $e_i = x^i$ for $i \in \[0; n+1 \]$. Assume $e_i$ are linearly dependent, meaning $\exists a_i$ s.t. $p(x) = \sum a_i e_i = 0$ for all $x$. But we know that polynomial of degree $n+1$ has at most $n+1$ roots unless it's zero polynomial. So in order to satisfy it for any point $a_i == 0$ for all $i$.
    Thus $x^i$ are linearly indepenent.
\end{proof}
    From the claim above, we have that $dim (P) = n+2$. We're given with a set of $n+1$ linear equations on cooeficients of the polynomial, thus the space of solutions is 1 dimensional. 

\subsection{
}
a) 

Clearly $0 \in T^{-1} (W_0)$, since $T(0) = 0 \in W_0$.
Let $a,b \in T^{-1} (W_0)$. Then $\exists T(a) = x, T(b) = y$. As $W_0$ is a abelian group, $x+y \in W_0$. Since $T$ is homomorphism $x+y = T(a) + T(b) = T(a + b) \in W_0$
Let $a \in T^{-1}(W_0)$ be an element in the preimage. $T(a) \in W_0$, and since $W_0$ is a abelian group $-T(a) \in W_0$. But since linear maps are homomorphisms, we have:
$-T(a) = T(-a) \in W_0$

Thus $T^{-1}$ is abelian subgroup. 

b) Let $f$ be restriction of $T$ on $X = T^{-1} W_0$. Then $f : X \to W_0$ is injective homomorphism and it induces an isomophism $f*: X/ker f^* \to W_0$. Notice that $\ker f^* = \ker f = \ker T$. Taking dim from both sides, we have:
$$ \dim (X) = \dim (W_0) + \dim (\ker (T))$$

\subsection{
}


    \end{document}